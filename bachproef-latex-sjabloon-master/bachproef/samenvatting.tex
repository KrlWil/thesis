%%=============================================================================
%% Samenvatting
%%=============================================================================

% TODO: De "abstract" of samenvatting is een kernachtige (~ 1 blz. voor een
% thesis) synthese van het document.
%
% Deze aspecten moeten zeker aan bod komen:
% - Context: waarom is dit werk belangrijk?
% - Nood: waarom moest dit onderzocht worden?
% - Taak: wat heb je precies gedaan?
% - Object: wat staat in dit document geschreven?
% - Resultaat: wat was het resultaat?
% - Conclusie: wat is/zijn de belangrijkste conclusie(s)?
% - Perspectief: blijven er nog vragen open die in de toekomst nog kunnen
%    onderzocht worden? Wat is een mogelijk vervolg voor jouw onderzoek?
%
% LET OP! Een samenvatting is GEEN voorwoord!

%%---------- Nederlandse samenvatting -----------------------------------------
%
% TODO: Als je je bachelorproef in het Engels schrijft, moet je eerst een
% Nederlandse samenvatting invoegen. Haal daarvoor onderstaande code uit
% commentaar.
% Wie zijn bachelorproef in het Nederlands schrijft, kan dit negeren, de inhoud
% wordt niet in het document ingevoegd.

\IfLanguageName{english}{%
\selectlanguage{dutch}
\chapter*{Samenvatting}
\lipsum[1-4]
\selectlanguage{english}
}{}

%%---------- Samenvatting -----------------------------------------------------
% De samenvatting in de hoofdtaal van het document

\chapter*{\IfLanguageName{dutch}{Samenvatting}{Abstract}}

% \lipsum[1-4]
Dit werk is een studie van het performantieverschil van de Meltdown en Spectre patches op een Linux webserver.
Om dit uit te testen werd er met Vagrant en Ansible een testopstelling opgezet, om gemakkelijk resultaten te genereren.
Het resultaat is data van een aantal benchmark en monitoring tools.
Aan de hand van t-testen worden de resultaten onderzocht, en er wordt gezien of ze statistisch significant zijn.

In dit document vind je achtergrond over hoe de Spectre/Meltdown aanvallen werken, en hoe risicobeperkende maatregelen de performantie kunnen beïnvloeden.
De resultaten duiden aan dat in de meeste applicaties er een significant verschil is.
Eerst werd er gezien of het processorgebruik tijdens een normale werklast veel verschilt.

Het antwoord was duidelijk, met een verschil van 18 procent.
De volgende test was op het vlak van responstijd, het verschil was hier redelijk, met 5 procent.
Redis, een in-memory database verloor maar 3 procent over allerlei andere testen.

Deze bachelorproef bevestigt dat de patches wel degelijk performantieverlies met zich meebrengen. 
Het performantieverlies voor een webserver is significant, doch niet rampzalig.
Dit is belangrijk voor webmasters die zich afvragen of ze nu al dan niet meer rekenkracht nodig hebben. De belangrijkste impact zal zijn op werkelijk grootschalige omgevingen zoals cloud-providers, waar zelfs een toename van een paar procenten van de belasting, extra hardware vereist.

De vraag is of de patches nog veel verbeterd kunnen worden, en wat de impact is op het vlak van energieverbruik, dat ook belangrijk is voor cloud-providers.






